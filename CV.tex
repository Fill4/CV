% !TEX TS-program = xelatex
% !TEX encoding = UTF-8

%%%%%%%%%%%%%%%%%%%%%%%%%%%%%%%%%%%%%%%%%%%%%%%%%%%%%%%%%%%%%%%%%
%% SIMPLE-RESUME-CV
%% <https://github.com/zachscrivena/simple-resume-cv>
%% This is free and unencumbered software released into the
%% public domain; see <http://unlicense.org> for details.
%%%%%%%%%%%%%%%%%%%%%%%%%%%%%%%%%%%%%%%%%%%%%%%%%%%%%%%%%%%%%%%%%

%%%%%%%%%%%%%%%%%%%%%%%%%%%%%%%%%%%%%%%%%%%%%%%%%%%%%%%%%%%%%%%%%
%% INSTRUCTIONS FOR COMPILING THIS DOCUMENT ("CV.tex")
%% TeX ---(XeLaTeX)---> PDF:
%%
%% Method 1: Use latexmk for fully automated document generation:
%%   latexmk -xelatex "CV.tex"
%%   (add the -pvc switch to automatically recompile on changes)
%%
%% Method 2: Use XeLaTeX directly:
%%   xelatex "CV.tex"
%%   (run multiple times to resolve cross-references if needed)
%%%%%%%%%%%%%%%%%%%%%%%%%%%%%%%%%%%%%%%%%%%%%%%%%%%%%%%%%%%%%%%%%

% \documentclass[a4paper,10pt,oneside]{article}
\documentclass[letterpaper,10pt,oneside]{article}

%%%%%%%%%%%%%%%%%%%%%%%%%%%%%%%%%%%%%%%%%%%%%%%%%%%%%%%%%%%%%%%%%
%% TYPESETTING OPTIONS.
%%%%%%%%%%%%%%%%%%%%%%%%%%%%%%%%%%%%%%%%%%%%%%%%%%%%%%%%%%%%%%%%%

\newcommand{\TypesetInNonStopMode}{1}
\newcommand{\TypesetInDraftMode}{0}

%%%%%%%%%%%%%%%%%%%%%%%%%%%%%%%%%%%%%%%%%%%%%%%%%%%%%%%%%%%%%%%%%
%% PREAMBLE.
%%%%%%%%%%%%%%%%%%%%%%%%%%%%%%%%%%%%%%%%%%%%%%%%%%%%%%%%%%%%%%%%%

%%%%%%%%%%%%%%%%%%%%%%%%%%%%%%%%%%%%%%%%%%%%%%%%%%%%%%%%%%%%%%%%%
%% SIMPLE-RESUME-CV
%% <https://github.com/zachscrivena/simple-resume-cv>
%% This is free and unencumbered software released into the
%% public domain; see <http://unlicense.org> for details.
%%%%%%%%%%%%%%%%%%%%%%%%%%%%%%%%%%%%%%%%%%%%%%%%%%%%%%%%%%%%%%%%%

% Run in non-stop mode.
\ifnum\TypesetInNonStopMode=1
\nonstopmode
\fi

% Geometry package for page margins.
\usepackage[
left=0.70in,
right=0.70in,
top=0.60in,
bottom=0.45in,
nohead,
includefoot]{geometry}

% PDF settings and properties.
\usepackage{hyperref}

% Long table for page layout.
\usepackage{longtable}

% Hyphenation: Disabled.
\usepackage[none]{hyphenat}

% Colors.
\usepackage[usenames]{color}
% \definecolor{MyDarkBlue}{RGB}{0,90,160}
% {\color{MyDarkBlue}This text is dark blue}

% Current date and time.
\usepackage[yyyymmdd,24hr]{datetime}
\renewcommand{\dateseparator}{-}
\settimeformat{xxivtime}
% {\today}~{\currenttime}

% Timestamp.
\newcommand{\Timestamp}{{\yyyymmdddate\today}~{\currenttime}}

% Abbreviations for months.
\newcommand{\LongMonth}[1]{%
\ifcase#1\relax
\or January%
\or February%
\or March%
\or April%
\or May%
\or June%
\or July%
\or August%
\or September%
\or October%
\or November%
\or December%
\fi}
\newcommand{\ShortMonth}[1]{%
\ifcase#1\relax
\or Jan%
\or Feb%
\or Mar%
\or Apr%
\or May%
\or Jun%
\or Jul%
\or Aug%
\or Sep%
\or Oct%
\or Nov%
\or Dec%
\fi}

% Select datestamp format.
\def\DatestampFormatSelection{2}

% Datestamp format: {yyyy}{MM}{dd} ---> yyyy-MM-dd (e.g., 2010-12-31).
\ifnum\DatestampFormatSelection=1
\newcommand{\DatestampYMD}[3]{\mbox{#1-#2-#3}}
\newcommand{\DatestampYM}[2]{\mbox{#1-#2}}
\newcommand{\DatestampY}[1]{#1}
\fi

% Datestamp format: {yyyy}{MM}{dd} ---> MMM yyyy (e.g., Dec 2010).
\ifnum\DatestampFormatSelection=2
\newcommand{\DatestampYMD}[3]{\mbox{\ShortMonth{#2} #1}}
\newcommand{\DatestampYM}[2]{\mbox{\ShortMonth{#2} #1}}
\newcommand{\DatestampY}[1]{#1}
\fi

% Datestamp format: {yyyy}{MM}{dd} ---> MMMM yyyy (e.g., December 2010).
\ifnum\DatestampFormatSelection=3
\newcommand{\DatestampYMD}[3]{\mbox{\LongMonth{#2} #1}}
\newcommand{\DatestampYM}[2]{\mbox{\LongMonth{#2} #1}}
\newcommand{\DatestampY}[1]{#1}
\fi

% Datestamp format: {yyyy}{MM}{dd} ---> yyyy (e.g., 2010).
\ifnum\DatestampFormatSelection=4
\newcommand{\DatestampYMD}[3]{#1}
\newcommand{\DatestampYM}[2]{#1}
\newcommand{\DatestampY}[1]{#1}
\fi

% XeLaTeX packages.
\usepackage{fontspec}
\defaultfontfeatures{Ligatures=TeX}
\usepackage{xunicode}
\usepackage{xltxtra}

% Font: Use "Tinos" as the main typeface (\textnormal{}, \normalfont).
% The "Tinos" fonts are released under the Apache License Version 2.0,
% and can be downloaded for free at <http://www.fontsquirrel.com/fonts/tinos>.
% Symbol table: <http://www.fileformat.info/info/unicode/font/tinos/grid.htm>
\setmainfont
[Path=./Fonts/Tinos/,
ItalicFont=Tinos-Italic,
BoldFont=Tinos-Bold,
BoldItalicFont=Tinos-BoldItalic]
{Tinos-Regular.ttf}

% Secondary font: "GNU FreeFont".
% The "GNU FreeFont" fonts are released under the
% GNU General Public License Version 3, and can be downloaded
% for free at <https://savannah.gnu.org/projects/freefont/>.
\newcommand{\UseSecondaryFont}{\fontspec
[Path=./Fonts/GNUFreeFont/,
ItalicFont=FreeSerifItalic,
BoldFont=FreeSerifBold,
BoldItalicFont=FreeSerifBoldItalic]
{FreeSerif.otf}}

% Sans-serif font: Changed to "Tinos".
\renewcommand{\sffamily}{\rmfamily}

% Typewriter (monospace) font: Changed to "Tinos".
\renewcommand{\ttfamily}{\rmfamily}

% Small caps font: Changed to "Tinos".
\renewcommand{\scshape}{\rmfamily}
%\renewcommand{\textsc}[1]{\textbf{\MakeUppercase{#1}}}

% Font styles.
\newcommand{\UseHeadingFont}{\normalfont}
\newcommand{\UseHeaderFooterFont}{\UseHeadingFont\fontsize{8.2pt}{9.5pt}\selectfont}
\newcommand{\UseNoteFont}{\UseHeadingFont\fontsize{8pt}{9.6pt}\selectfont}
\newcommand{\UseTitleFont}{\UseHeadingFont\fontsize{28pt}{33.6pt}\selectfont\bfseries}
\newcommand{\UseSubTitleFont}{\normalfont\fontsize{8.6pt}{10.3pt}\selectfont}
\newcommand{\UseSectionFont}{\UseHeadingFont\fontsize{9pt}{11pt}\selectfont\bfseries}
\newcommand{\UseSubSectionFont}{\UseHeadingFont\fontsize{8.6pt}{10.3pt}\selectfont\bfseries}
\newcommand{\UseDetailFont}{\normalfont\fontsize{8.6pt}{10.3pt}\selectfont}

% Symbols (unicode).
\newcommand{\BulletSymbol}{{\normalfont\fontsize{6.5}{7.8}\selectfont\raisebox{0.17em}{\char"25A0}}}
\newcommand{\SubBulletSymbol}{{\normalfont\fontsize{6}{7.2}\selectfont\raisebox{0.17em}{\char"25CF}}}
\newcommand{\TildeSymbol}{{\normalfont\char"007E}}

% Headers and footers: Blank header, page number in footer.
\ifnum\TypesetInDraftMode=0
\newcommand{\HeaderText}{}
\newcommand{\FooterText}{\UseHeaderFooterFont\hfill%
{Page}~{\thepage}~of~\pageref{LastPage}%
\hfill}
\else
\newcommand{\HeaderText}{}
\newcommand{\FooterText}{\UseHeaderFooterFont%
\hphantom{DRAFT~\Timestamp}\hfill%
{Page}~{\thepage}~of~\pageref{LastPage}%
\hfill{\color{red}DRAFT~\Timestamp}}
\fi

\makeatletter
\def\ps@plain{%
\def\@oddhead{\HeaderText}%
\def\@evenhead{\HeaderText}%
\def\@oddfoot{\FooterText}%
\def\@evenfoot{\FooterText}}
\makeatother

\pagestyle{plain}

% Paragraph style.
\setlength{\parindent}{0in} % No indentation at the beginning of each paragraph.
\setlength{\parskip}{0in} % No vertical space between paragraphs.

% Footnotes: Use symbols instead of numbers for labels.
\renewcommand{\thefootnote}{\fnsymbol{footnote}}

% Macro: title (name).
\renewcommand{\title}[1]{%
\pdfbookmark[1]{#1}{#1}%
\par\begin{center}%
\par\UseTitleFont%
{#1}%
\par\end{center}%
\par\vspace{-1.75em}\par}

% Macro: subtitle (personal information below name).
\newenvironment{subtitle}
{\par\begin{center}%
\par\UseSubTitleFont}
{\par\end{center}\par}

% Macro: body (rest of the document).
\newenvironment{body}
{\par\vspace{-1em}\par
\begin{longtable}{p{0.15\textwidth}p{0.80\textwidth}}}
{\par\end{longtable}\par}

% Macro: section (new section for Education, Research Experience, etc.).
\renewcommand{\section}[3]{\\[-1em]\pdfbookmark[2]{#2}{#3}\\%
{\UseSectionFont\raggedright\MakeUppercase{#1}}%
&}

% Macro: subsection.
\renewcommand{\subsection}[3]{\par~\vskip-\baselineskip%
\pdfbookmark[3]{#2}{#3}\par%
{\UseSubSectionFont\raggedright\MakeUppercase{#1}}%
\vspace{0.225\baselineskip}}

% Macro: BigGap, BigGapNoBreak (big vertical gap between items in the same section).
\newcommand{\BigGap}{\\[-1.75mm]~&}
\newcommand{\BigGapNoBreak}{\par\vspace{2.45mm}\par}

% Macro: Gap, GapNoBreak (vertical gap between items in the same section).
\newcommand{\Gap}{\\[-3.5mm]~&}
\newcommand{\GapNoBreak}{\par\vspace{0.7mm}\par}

% Macro: detail (text in smaller font under an item).
\newenvironment{detail}
{\par\begingroup\UseDetailFont}
{\par\endgroup\par}

% Macro: BulletItem.
\newsavebox{\BulletItemIndentation}
\newlength{\BulletItemIndentationWidth}
\newcommand{\BulletItem}{\par%
\savebox{\BulletItemIndentation}{\hspace{1.5mm}\BulletSymbol\hspace{1.25mm}}%
\settowidth{\BulletItemIndentationWidth}{\usebox{\BulletItemIndentation}}%
\noindent\hangafter=1\hangindent=\BulletItemIndentationWidth\ignorespaces%
\usebox{\BulletItemIndentation}\ignorespaces}

% Macro: SubBulletItem.
\newsavebox{\SubBulletItemIndentation}
\newlength{\SubBulletItemIndentationWidth}
\newcommand{\SubBulletItem}{\par%
\savebox{\SubBulletItemIndentation}{\hspace{5.6mm}\SubBulletSymbol\hspace{1.25mm}}%
\settowidth{\SubBulletItemIndentationWidth}{\usebox{\SubBulletItemIndentation}}%
\noindent\hangafter=1\hangindent=\SubBulletItemIndentationWidth\ignorespaces%
\usebox{\SubBulletItemIndentation}\ignorespaces}

% Macro: Item.
\newsavebox{\ItemIndentation}
\newlength{\ItemIndentationWidth}
\newcommand{\Item}{\par%
\savebox{\ItemIndentation}{\hphantom{\hspace{1.5mm}\BulletSymbol\hspace{1.25mm}}}%
\settowidth{\ItemIndentationWidth}{\usebox{\ItemIndentation}}%
\noindent\hangafter=1\hangindent=\ItemIndentationWidth\ignorespaces%
\usebox{\ItemIndentation}\ignorespaces}

% Macro: SubItem.
\newsavebox{\SubItemIndentation}
\newlength{\SubItemIndentationWidth}
\newcommand{\SubItem}{\par%
\savebox{\SubItemIndentation}{\hphantom{\hspace{1.5mm}\BulletSymbol\hspace{1.25mm}}}%
\settowidth{\SubItemIndentationWidth}{\usebox{\SubItemIndentation}}%
\noindent\hangafter=1\hangindent=\SubItemIndentationWidth\ignorespaces%
\usebox{\SubItemIndentation}\ignorespaces}

% Macro: NumberedItem.
\newsavebox{\NumberedItemIndentation}
\newlength{\NumberedItemIndentationWidth}
\newcommand{\NumberedItem}[1]{\par%
\savebox{\NumberedItemIndentation}{{#1}\hspace{2.3mm}}%
\settowidth{\NumberedItemIndentationWidth}{\usebox{\NumberedItemIndentation}}%
\noindent\hangafter=1\hangindent=\NumberedItemIndentationWidth\ignorespaces%
\usebox{\NumberedItemIndentation}\ignorespaces}

% Macro: CharSpace (for aligning single-digit numbers).
\newlength{\CharWidth}
\newcommand{\CharSpace}{\settowidth{\CharWidth}{8}\hspace{\CharWidth}}

% Macro: hide.
\newcommand{\hide}[1]{}


% CV Info (to be customized).
\newcommand{\CVAuthor}{Filipe Pereira}
\newcommand{\CVTitle}{Curriculum Vit\ae}
\newcommand{\CVNote}{CV}
\newcommand{\CVWebpage}{}

% PDF settings and properties.
\hypersetup{
	pdftitle={\CVTitle},
	pdfauthor={\CVAuthor},
	pdfsubject={\CVWebpage},
	pdfcreator={XeLaTeX},
	pdfproducer={},
	pdfkeywords={},
	pdfpagemode={},
	bookmarks=true,
	unicode=true,
	bookmarksopen=true,
	pdfstartview=FitH,
	pdfpagelayout=OneColumn,
	pdfpagemode=UseOutlines,
	hidelinks,
	breaklinks,
	% Color settings
	colorlinks = true,
    linkcolor = blue,
    urlcolor  = blue,
    citecolor = blue,
    anchorcolor = blue
}

% Shorthand.
\newcommand{\CodeCommand}[1]{\mbox{\textbf{\textbackslash{#1}}}}

%%%%%%%%%%%%%%%%%%%%%%%%%%%%%%%%%%%%%%%%%%%%%%%%%%%%%%%%%%%%%%%%%
%% ACTUAL DOCUMENT.
%%%%%%%%%%%%%%%%%%%%%%%%%%%%%%%%%%%%%%%%%%%%%%%%%%%%%%%%%%%%%%%%%

\begin{document}

%%%%%%%%%%%%%%%
% TITLE BLOCK %
%%%%%%%%%%%%%%%

\title{\CVAuthor}
%\subtitle{\CVTitle}

\begin{subtitle}
%{Rua Dr. Orlando de Oliveira, nº11, 2ºDto, 3800-004, Aveiro, Portugal}

\par
\GapNoBreak
\href{mailto:filipe.pereira@astro.up.pt}{filipe.pereira@astro.up.pt} \\
Github: \href{https://github.com/Fill4}{github.com/Fill4} \\ 
Orcid: \href{https://orcid.org/0000-0002-2157-7146}{0000-0002-2157-7146}
%\,\SubBulletSymbol\,
%+351 913 231 818
%\,\SubBulletSymbol\,
%\href{\CVWebpage}
%{\CVWebpage}
\end{subtitle}

\begin{body}

%%%%%%%%%%%%%%%
%% EDUCATION %%
%%%%%%%%%%%%%%%

\section
{Education}
{Education}
{PDF:Education}

\textbf{PhD in Astronomy}, Faculty of Sciences, University of Porto
\hfill
% \DatestampYM{2016}{10} -- 
\DatestampYM{2021}{12}
\begin{detail}
\BulletItem
Advisors: Tiago Compante, Margarida Cunha, Nuno Santos
\BulletItem
Thesis: \href{https://hdl.handle.net/10216/138552}{Detection and characterization of planets orbiting oscillating red-giant stars with NASA's TESS mission}
\end{detail}

\GapNoBreak
\textbf{MSc in Astronomy}, Faculty of Sciences, University of Porto
\hfill
% \DatestampYM{2014}{10} -- 
\DatestampYM{2016}{09}
\begin{detail}
\BulletItem
Advisor: Mário João Monteiro
\BulletItem
Thesis: \href{https://repositorio-aberto.up.pt/handle/10216/90991}{Development of automatic tools for measuring acoustic glitches in seismic data of solar-type stars}
\end{detail}

%\begin{detail}
%\SubBulletItem
%Thesis: Development of automatic tools for measuring acoustic glitches in seismic data of solar-type stars
%\SubBulletItem
%Supervisor:
%Professor Mário João Monteiro
%\SubBulletItem
%Research areas: Asteroseismology, acoustic glicthes, Fortran
%\end{detail}

\GapNoBreak
\textbf{BSc in Astronomy}, Faculty of Sciences, University of Porto
\hfill
% \DatestampYM{2011}{09} -- 
\DatestampYM{2014}{09}


%%%%%%%%%%%%%%%%%%
%% PUBLICATIONS %%
%%%%%%%%%%%%%%%%%%

\section
{Refereed Publications}
{Refereed Publications}
{PDF:Publications}

See list in \href{https://ui.adsabs.harvard.edu/public-libraries/gKWEYjBTR7KtFkRn00r-pw}{ADS} 

\GapNoBreak
\subsection
{Journals}
{Journals}
{PDF:Journals}

% \NumberedItem{1.}
\BulletItem
Grunblatt S., Saunders, N., et al. including \textbf{Pereira, F.} \newline
\textit{TESS Giants Transiting Giants II: The hottest Jupiters orbiting evolved stars}. \newline
2022, Accepted for publication in AAS
%2022, AAS, --, --, 
%DOI: \href{}{--}, 
%arXiv: \href{}{--}.

% \NumberedItem{2.}
\BulletItem
Lund, M. N., Handberg, R., et al. including \textbf{Pereira, F.} \newline
\textit{TESS Data for Asteroseismology: Light-curve Systematics Correction}. \newline
2021, ApJS, 257, 53,
DOI: \href{https://doi.org/10.3847/1538-4365/ac214a}{10.3847/1538-4365/ac214a}, 
arXiv: \href{https://arxiv.org/abs/2108.11780}{2108.11780}.

% \NumberedItem{3.}
\BulletItem
Lillo-box, J. Ribas, Á., et al. including \textbf{Pereira, F.} \newline
\textit{Uncovering the ultimate planet impostor. An eclipsing brown dwarf in a hierarchical triple with two evolved stars}. \newline
2021, A\&A, 653, A40,
DOI: \href{https://doi.org/10.1051/0004-6361/202141158}{10.1051/0004-6361/202141158}, 
arXiv: \href{https://arxiv.org/abs/2106.05011}{2106.05011}.

% \NumberedItem{4.}
\BulletItem
Silva Aguirre, V., Stello, D., et al. including \textbf{Pereira, F.} \newline
\textit{Detection and Characterization of Oscillating Red Giants: First Results from the TESS Satellite}. \newline
2020, ApJL, 889, L34,
DOI: \href{https://doi.org/10.3847/2041-8213/ab6443}{10.3847/2041-8213/ab6443}, 
arXiv: \href{https://arxiv.org/abs/1912.07604}{1912.07604}.

% \NumberedItem{5.}
\BulletItem
Barros S.C.C., Demangeon, O., et al. including \textbf{Pereira, F.} \newline
\textit{Improving transit characterisation with Gaussian process modelling of stellar variability}. \newline
2020, A\&A, 634, A75,
DOI: \href{https://doi.org/10.1051/0004-6361/201936086}{10.1051/0004-6361/201936086}, 
arXiv: \href{https://arxiv.org/abs/2001.07975}{2001.07975}.

% \NumberedItem{6.}
\BulletItem
\textbf{Pereira, F.}, Campante, T. L., Cunha, M. S. et al. \newline
\textit{Gaussian process modelling of granulation and oscillations in red giant stars}. \newline
2019, MNRAS, 489, 5764,
DOI: \href{https://doi.org/10.1093/mnras/stz2405}{10.1093/mnras/stz2405}, 
arXiv: \href{https://arxiv.org/abs/1908.10662}{1908.10662}.

% \NumberedItem{7.}
\BulletItem
Campante, T. L., Corsaro, E., et al. including \textbf{Pereira, F.} \newline
\textit{TESS Asteroseismology of the Known Red-giant Host Stars HD 212771 and HD 203949}. \newline
2019, ApJ, 885, 31,
DOI: \href{https://doi.org/10.3847/1538-4357/ab44a8}{10.3847/1538-4357/ab44a8}, 
arXiv: \href{https://arxiv.org/abs/1909.05961}{1909.05961}.

% \NumberedItem{8.}
\BulletItem
Huber, D., Chaplin, W. J., et al. including \textbf{Pereira, F.} \newline
\textit{A Hot Saturn Orbiting an Oscillating Late Subgiant Discovered by TESS}. \newline
2019, AJ, 157, 245,
DOI: \href{https://doi.org/10.3847/1538-3881/ab1488}{10.3847/1538-3881/ab1488}, 
arXiv: \href{https://arxiv.org/abs/1901.01643}{1901.01643}.

\BigGap
\subsection
{Proceedings}
{Proceedings}
{PDF:Proceedings}

% \NumberedItem{1.}
\BulletItem
\textbf{Pereira, F.}, Faria, J. P. S. and Monteiro, M. J. P. F. G. \newline
\textit{SIGS - Seismic Inferences for Glitches in Stars}. \newline
2017, EPJWC, 160, 01015,
DOI: \href{https://doi.org/10.1051/epjconf/201716001015}{10.1051/epjconf/201716001015}, 
arXiv: \href{https://arxiv.org/abs/1703.04828}{1703.04828}.

%%%%%%%%%%%%%%%%%%%%%
%% TALKS / POSTERS %%
%%%%%%%%%%%%%%%%%%%%%

\section
{Talks/Posters}
{Talks/Posters}
{PDF:TalksPosters}

\subsection
{Talks}
{Talks}
{PDF:Talks}

% \NumberedItem{1.}
\BulletItem
\emph{Using Gaussian Processes to model stellar granulation and oscillations in red-giant stars}. \newline
\href{http://conferences.au.dk/tasc4/}{TASC4/KASC11 Workshop}. July 2018, Aarhus, Denmark

\GapNoBreak

% \NumberedItem{2.}
\BulletItem
\emph{Using Gaussian Processes to model stellar granulation and oscillations in red-giant stars}. \newline
\href{https://sites.google.com/view/ras-evolsystems/home}{RAS Specialist Discussion Meeting}. March 2018, London, UK


\BigGap
\subsection
{Posters}
{Posters}
{PDF:Posters}

% \NumberedItem{1.}
\BulletItem
\emph{Gaussian process modelling of TESS light curves in the presence of stellar variability and transits}. \newline
\href{https://web.mit.edu/tasc5/}{TASC5/KASC12 Workshop}. July 2019, Boston, USA

\GapNoBreak

% \NumberedItem{2.}
\BulletItem
\emph{Tessting gaussian process regression for red-giant granulation modelling}. \newline
\href{https://www.tasc3kasc10.com/}{TASC3/KASC10 Workshop}. July 2017, Birmingham, UK

\GapNoBreak

% \NumberedItem{3.}
\BulletItem
\emph{SIGS - Seismic Inferences for Glitches in Stars}. \newline
\href{http://www.iastro.pt/research/conferences/spacetk16/}{TASC2/KASC9 Workshop}. July 2016, Azores, Portugal

    
% \BigGap
% \subsection
% {Reports}
% {Reports}
% {PDF:Reports}

% \NumberedItem{1.}
% Msc Grant Final Report
% \NumberedItem{2.}
% PEEC Project Report
% \NumberedItem{3.}
% Research Seminar in Astronomy Project Report


%%%%%%%%%%%%%%%%%%%%%
%% ACADEMIC AWARDS %%
%%%%%%%%%%%%%%%%%%%%%

\section
{Grants}
{Grants}
{PDF:Grants}

\BulletItem
PhD Grant - FCT ($\sim$60,000 EUR) \hfill 
\DatestampY{2017} -- \DatestampY{2021}

\GapNoBreak
\BulletItem
Research Grant - FCT ($\sim$12,000 EUR) \hfill 
\DatestampY{2017} -- \DatestampY{2017}

\GapNoBreak
\BulletItem
MSc Grant - SPACEINN ($\sim$7,000 EUR) \hfill 
\DatestampY{2015} -- \DatestampY{2016}

% Development of automatic tools for measuring acoustic glitches in seismic 
% \hfill
% \DatestampY{2015}{1 \par
%  data of solar-type stars
% \BulletItem
% Project: SPACEINN (FP7-SPACE-2012-312844)
% \BulletItem
% Advisor: Dr. Mário João Monteiro
% \BulletItem
% Institution: Centro de Astrofísica da Universidade do Porto (CAUP)
% \BulletItem
% Grant: \textasciitilde 7000 EUR

% \BigGap
% ​Characterizing and modelling red giants
% \hfill
% \DatestampYMD{2017}{01}{01} --
% \DatestampYMD{2017}{12}{31}
% \BulletItem
% Project: ​CIAAUP-09/2016-BI
% \BulletItem
% Advisor: Dr. Margarida Cunha
% \BulletItem
% Institution: Centro de Astrofísica da Universidade do Porto (CAUP)
% \BulletItem
% Grant: \textasciitilde 12000 EUR


%%%%%%%%%%%%%%%%%%%%%%%%%%%%%%%%%
%% OTHER CURRICULAR ACTIVITIES %%
%%%%%%%%%%%%%%%%%%%%%%%%%%%%%%%%%

% \section
% {Other Curricular Activities}
% {Other Curricular Activities}
% {PDF:OtherCurricularActivities}

% \href{http://www.iastro.pt/research/conferences/faial2016/}
% {\textbf{IVth Azores International Advanced School in Space Sciences}},
% Azores, Portugal
% \hfill
% \DatestampYM{2016}{07}
%\GapNoBreak
%\BulletItem
%Completed various courses totaling 49 hours.
%\BulletItem
%Included courses in: Stellar Modelling, Theory of Stellar Oscillations, Data Analysis in Asteroseismology, Exoplanetary Science, Analysis of Photometric Time Series

% \GapNoBreak
% {\textbf{Machine Learning Course from Stanford University}}
% \hfill
% \DatestampYMD{2015}{09}{15} --
% \DatestampYMD{2015}{12}{07}


%%%%%%%%%%%%%%%%%%%%%%%%%%%%
%% ACADEMIC RESEARCH WORK %%
%%%%%%%%%%%%%%%%%%%%%%%%%%%%

% \section
% {Academic Research Work}
% {Academic Research Work}
% {PDF:AcademicResearchWork}

% {\textbf{Research Seminar in Astronomy}}
% \hfill
% \DatestampYMD{2017}{07}{15} --
% \DatestampYMD{2016}{10}{01}
% \begin{detail}
% \SubBulletItem
% Project: Testing gaussian process regression for red-giant granulation modelling
% \SubBulletItem
% Supervisor:
% Dr. Margarida Cunha
% \SubBulletItem
% Research areas: red-giant stars, granulation, gaussian processes, python
% \end{detail}

% {\textbf{Msc Dissertation}}
% \hfill
% \DatestampYMD{2016}{10}{15} --
% \DatestampYMD{2015}{10}{01}
% \begin{detail}
% \SubBulletItem
% Thesis: Development of automatic tools for measuring acoustic glitches in seismic data of solar-type stars
% \SubBulletItem
% Supervisor:
% Dr. Mário João Monteiro
% \SubBulletItem
% Research areas: Asteroseismology, acoustic glicthes, Fortran
% \end{detail}

% \GapNoBreak
% {\textbf{Undergradute Research Project}}
% \hfill
% \DatestampYMD{2014}{02}{15} --
% \DatestampYMD{2014}{07}{15}
% \begin{detail}
% \SubBulletItem
% Project:
% Study of the mizimization function in the ARES+MOOG procedure
% \SubBulletItem
% Supervisor:
% Dr. Sérgio Sousa
% \SubBulletItem
% Research areas: Spectroscopic Parameters, iron line abundances, Python
% \end{detail}


%%%%%%%%%%%%%%%%%%%%%%%%%
%% RESEARCH EXPERIENCE %%
%%%%%%%%%%%%%%%%%%%%%%%%%

%\section
%{Research Experience}
%{Research Experience}
%{PDF:ResearchExperience}


%%%%%%%%%%%%%%
%% TEACHING %%
%%%%%%%%%%%%%%

\section
{Teaching}
{Teaching}
{PDF:Teaching}

\BulletItem
Escola de Verão de Física 2019 (Physics Summer School). Faculty of Sciences, University of Porto
% \hfill \DatestampYM{2019}{09}
\GapNoBreak
\SubBulletItem
Supervised and lectured a project about "The energy of stars" for a class of 5 high-school students

\GapNoBreak
\BulletItem
Escola de Verão de Física 2018 (Physics Summer School). Faculty of Sciences, University of Porto
% \hfill \DatestampYM{2018}{09}
\GapNoBreak
\SubBulletItem
Supervised and lectured a project about "The energy of stars" for a class of 5 high-school students

\GapNoBreak
\BulletItem
Escola de Verão de Física 2017 (Physics Summer School). Faculty of Sciences, University of Porto
% \hfill \DatestampYM{2017}{09}
\GapNoBreak
\SubBulletItem
Supervised and lectured a project about "The energy of stars" for a class of 5 high-school students




%%%%%%%%%%%%%%%%%%%%%%%%%%%%%%%%%%%%%%%%%%%%
%% SCIENTIFIC ACTIVITIES                  %%
%%%%%%%%%%%%%%%%%%%%%%%%%%%%%%%%%%%%%%%%%%%%

\section
{Scientific Participation}
{Scientific Participation}
{PDF:ScietificParticipation}

% {\textbf{Research Internship}}
% \hfill
% \DatestampYMD{2016}{10}{15} -- \DatestampYMD{2017}{09}{30}
% \begin{detail}
% \SubBulletItem
% Project:
% ​Characterizing and modelling red giants
% \SubBulletItem
% Supervisor:
% Dr. Margarida Cunha
% \SubBulletItem
% Research areas: gaussian processes, grid-based modelling, red-giant stars
% \end{detail}

% \BigGap
% {\textbf{PEEC (Extra-Curricular Research Project)}}
% \hfill
% \DatestampYMD{2013}{10}{15} --
% \DatestampYMD{2015}{07}{15}
% \begin{detail}
% \SubBulletItem
% Project:
% Testing the applicability of using titanium line abundances to determine spectroscopic stellar parameters, especially surface gravity, using the ARES and MOOG tools
% \SubBulletItem
% Supervisor:
% Dr. Sérgio Sousa
% \SubBulletItem
% Research areas: Spectroscopic Parameters, titanium line abundances, Python
% \end{detail}

\subsection
{Conferences}
{Conferences}
{PDF:Conferences}

\GapNoBreak
\BulletItem
\href{http://www.iastro.pt/research/conferences/8th-imas/}{8th Iberian Meeting on Asteroseismology}. \DatestampY{2021}, Online

\GapNoBreak
\BulletItem
\href{https://7thimas.cab.inta-csic.es/main/index.php}{7th Iberian Meeting on Asteroseismology}. \DatestampY{2020}, Online

\GapNoBreak
\BulletItem
\href{https://web.mit.edu/tasc5/}{TASC5/KASC12 Workshop}. \DatestampY{2019}, Boston, USA
% \hfill \DatestampYM{2019}{07}

\GapNoBreak
\BulletItem
\href{http://conferences.au.dk/tasc4/}{TASC4/KASC11 Workshop}. \DatestampY{2018}, Aarhus, Denmark
% \hfill \DatestampYM{2018}{07}

\GapNoBreak
\BulletItem
\href{https://sites.google.com/view/ras-evolsystems/home}{RAS Specialist Discussion Meeting}. \DatestampY{2017}, London, UK

\GapNoBreak
\BulletItem
\href{https://www.tasc3kasc10.com/}{TASC3/KASC10 Workshop}. \DatestampY{2017}, Birmingham, UK
% \hfill \DatestampYM{2017}{07}

\GapNoBreak
\BulletItem
\href{http://www.iastro.pt/research/conferences/spacetk16/}{TASC2/KASC9 Workshop}. \DatestampY{2016}, Azores, Portugal
% \hfill \DatestampYM{2016}{07}

\BigGap
\subsection
{Workshops}
{Workshops}
{PDF:Workshops}

\GapNoBreak
\BulletItem
T'DA8: 8th TESS Data for Asteroseismology workshop. \DatestampY{2019}, Aarhus, Denmark
% \hfill \DatestampYM{2019}{01}

\GapNoBreak
\BulletItem
T'DA6: 6th TESS Data for Asteroseismology workshop. \DatestampY{2018}, Leuven, Belgium
% \hfill \DatestampYM{2018}{11}

\GapNoBreak
\BulletItem
T'DA4: 4th TESS Data for Asteroseismology workshop. \DatestampY{2018}, Aarhus, Denmark
% \hfill \DatestampYM{2018}{07}

\GapNoBreak
\BulletItem
T'DA3: 3rd TESS Data for Asteroseismology workshop. \DatestampY{2017}, Leuven, Belgium
% \hfill \DatestampYM{2017}{12}

\GapNoBreak
\BulletItem
T'DA2: 2nd TESS Data for Asteroseismology workshop. \DatestampY{2017}, Aarhus, Denmark
% \hfill \DatestampYM{2017}{04}

\GapNoBreak
\BulletItem
T'DA1: 1st TESS Data for Asteroseismology workshop. \DatestampY{2016}, Birmingham, England
% \hfill \DatestampYM{2016}{10}

\GapNoBreak
\BulletItem
IVth Azores International Advanced School in Space Sciences. \DatestampY{2016}, Azores, Portugal
% \hfill \DatestampYM{2016}{07}

% \BigGap
% \textbf{Member of the IA Stars \& Planets Research Team}
% \GapNoBreak
% \BulletItem
% Participate in bi-weekly meetings with the Asteroseismology Team
% \GapNoBreak
% \BulletItem
% Participate in monthly meetings with the Exoplanets Team

% \BigGap
% \textbf{Member of CAUP (Center for Astrophysics of the University of Porto)}
% \GapNoBreak
% \BulletItem
% Attend regular seminar in various Astronomy topics presented by both local and visiting researchers


%%%%%%%%%%%%%%%%%%%%%%%%%%%%%%%%%%%%%%%%%%%%
%% SCIENTIFIC AFFILIATIONS                %%
%%%%%%%%%%%%%%%%%%%%%%%%%%%%%%%%%%%%%%%%%%%%

% \section
% {Scientific Affiliations}
% {Scientific Affiliations}
% {PDF:ScientificAffiliations}

% \href{http://www.sp-astronomia.pt/}
% {\textbf{Sociedade Portuguesa de Astronomia}},
% Portugal

% \GapNoBreak
% \BulletItem
% Member
% \hfill
% \DatestampY{2012} --
% Present

%%%%%%%%%%%%%%%
%% LANGUAGES %%
%%%%%%%%%%%%%%%

% \section
% {Languages}
% {Languages}
% {PDF:Languages}

% \BulletItem
% Portuguese: Native language.

% \GapNoBreak
% \BulletItem
% English: Fluent (speaking, reading, writing), C1 Level.

% \GapNoBreak
% \BulletItem
% Spanish: Intermediate (reading, speaking); basic (writing).

% %%%%%%%%%%%%
% %% SKILLS %%
% %%%%%%%%%%%%

\section
{Programming}
{Programming}
{PDF:Programming}

\subsection
{Languages}
{Languages}
{PDF:Languages}

Python, C++/C, Fortran, Bash, \LaTeX

\BigGap
\subsection
{Software developed}
{Software developed}
{PDF:SoftwareDeveloped}

\BulletItem
\href{https://github.com/Fill4/gptransits}{gptransits}: 
Simultaneous characterization of planetary transits and stellar signals in the time domain using Gaussian processes 
(\href{https://doi.org/10.1093/mnras/stz2405}{Pereira et al. 2019}).

\GapNoBreak
\BulletItem
\href{https://github.com/tasoc/corrections}{corrections}: 
TESS Asteroseismic Science Operations Center (TASOC) corrections module for the TESS asteroseismic data processing pipeline 
(contributed, \href{https://doi.org/10.3847/1538-4365/ac214a}{Lund et al. 2021}).

\GapNoBreak
\BulletItem
\href{https://github.com/Fill4/sigs_freq}{sigs\_freq} \textbackslash \ \href{https://github.com/Fill4/sigs_diff}{sigs\_diff}: 
Fit acoustic glitches (base of the convective zone and helium second ionization zone) in low-degree p-mode frequencies (sigs\_freq) or in the second differences of the frequencies (sigs\_diff) using the PIKAIA genetic algorithm 
(\href{https://doi.org/10.1051/epjconf/201716001015}{Pereira et al. 2017}).

% \GapNoBreak
% \BulletItem
% Intermediate Knowledge: {\LaTeX}, Python, MATLAB, Fortran

%%%%%%%%%%%%%%%
%% INTERESTS %%
%%%%%%%%%%%%%%%

% \section
% {Interests}
% {Interests}
% {PDF:Interests}

% Computer Science, Cinema, Basketball, Voleyball

\end{body}
\label{LastPage}
\end{document}
