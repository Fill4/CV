% !TEX TS-program = xelatex
% !TEX encoding = UTF-8

%%%%%%%%%%%%%%%%%%%%%%%%%%%%%%%%%%%%%%%%%%%%%%%%%%%%%%%%%%%%%%%%%
%% SIMPLE-RESUME-CV
%% <https://github.com/zachscrivena/simple-resume-cv>
%% This is free and unencumbered software released into the
%% public domain; see <http://unlicense.org> for details.
%%%%%%%%%%%%%%%%%%%%%%%%%%%%%%%%%%%%%%%%%%%%%%%%%%%%%%%%%%%%%%%%%

%%%%%%%%%%%%%%%%%%%%%%%%%%%%%%%%%%%%%%%%%%%%%%%%%%%%%%%%%%%%%%%%%
%% INSTRUCTIONS FOR COMPILING THIS DOCUMENT ("CV.tex")
%% TeX ---(XeLaTeX)---> PDF:
%%
%% Method 1: Use latexmk for fully automated document generation:
%%   latexmk -xelatex "CV.tex"
%%   (add the -pvc switch to automatically recompile on changes)
%%
%% Method 2: Use XeLaTeX directly:
%%   xelatex "CV.tex"
%%   (run multiple times to resolve cross-references if needed)
%%%%%%%%%%%%%%%%%%%%%%%%%%%%%%%%%%%%%%%%%%%%%%%%%%%%%%%%%%%%%%%%%

% \documentclass[a4paper,10pt,oneside]{article}
\documentclass[letterpaper,10pt,oneside]{article}

%%%%%%%%%%%%%%%%%%%%%%%%%%%%%%%%%%%%%%%%%%%%%%%%%%%%%%%%%%%%%%%%%
%% TYPESETTING OPTIONS.
%%%%%%%%%%%%%%%%%%%%%%%%%%%%%%%%%%%%%%%%%%%%%%%%%%%%%%%%%%%%%%%%%

\newcommand{\TypesetInNonStopMode}{1}
\newcommand{\TypesetInDraftMode}{0}

%%%%%%%%%%%%%%%%%%%%%%%%%%%%%%%%%%%%%%%%%%%%%%%%%%%%%%%%%%%%%%%%%
%% PREAMBLE.
%%%%%%%%%%%%%%%%%%%%%%%%%%%%%%%%%%%%%%%%%%%%%%%%%%%%%%%%%%%%%%%%%

\input{CV-Preamble.tex}

% CV Info (to be customized).
\newcommand{\CVAuthor}{Filipe Pereira}
\newcommand{\CVTitle}{Curriculum Vit\ae}
\newcommand{\CVNote}{CV compiled on {\today} for Acme Corporation}
\newcommand{\CVWebpage}{http://www.example.com/johndoe}

% PDF settings and properties.
\hypersetup{
pdftitle={\CVTitle},
pdfauthor={\CVAuthor},
pdfsubject={\CVWebpage},
pdfcreator={XeLaTeX},
pdfproducer={},
pdfkeywords={},
pdfpagemode={},
bookmarks=true,
unicode=true,
bookmarksopen=true,
pdfstartview=FitH,
pdfpagelayout=OneColumn,
pdfpagemode=UseOutlines,
hidelinks,
breaklinks}

% Shorthand.
\newcommand{\CodeCommand}[1]{\mbox{\textbf{\textbackslash{#1}}}}

%%%%%%%%%%%%%%%%%%%%%%%%%%%%%%%%%%%%%%%%%%%%%%%%%%%%%%%%%%%%%%%%%
%% ACTUAL DOCUMENT.
%%%%%%%%%%%%%%%%%%%%%%%%%%%%%%%%%%%%%%%%%%%%%%%%%%%%%%%%%%%%%%%%%

\begin{document}

%%%%%%%%%%%%%%%
% TITLE BLOCK %
%%%%%%%%%%%%%%%

\title{\CVAuthor}
\subtitle{\CVTitle}

\begin{subtitle}
%{Rua Dr. Orlando de Oliveira, nº11, 2ºDto, 3800-004, Aveiro, Portugal}
\par
\href{mailto:filipe.pereira@astro.up.pt}
{filipe.pereira@astro.up.pt}
%\,\SubBulletSymbol\,
%+351 913 231 818
%\,\SubBulletSymbol\,
%\href{\CVWebpage}
%{\CVWebpage}
\end{subtitle}

\begin{body}

%%%%%%%%%%%%%%%
%% EDUCATION %%
%%%%%%%%%%%%%%%

\section
{Education}
{Education}
{PDF:Education}

{\textbf{Faculdade de Ciências da Universidade do Porto}},
Porto, Portugal

\GapNoBreak
\BulletItem
M.Sc. in Astronomy
\hfill
\DatestampYMD{2014}{10}{15} --
\DatestampYMD{2016}{10}{15}
\begin{detail}
\SubBulletItem
Thesis: Development of automatic tools for measuring acoustic glitches in seismic data of solar-type stars
\SubBulletItem
Supervisor:
Professor Mário João Monteiro
\SubBulletItem
Research areas: Asteroseismology, acoustic glicthes, Fortran
\end{detail}

\GapNoBreak
\BulletItem
Bachelor's Degree in Astronomy
\hfill
\DatestampYMD{2011}{09}{15} --
\DatestampYMD{2014}{09}{15}

%%%%%%%%%%%%%%%%%%%%%%%%%
%% RESEARCH EXPERIENCE %%
%%%%%%%%%%%%%%%%%%%%%%%%%

\section
{Research Experience}
{Research Experience}
{PDF:ResearchExperience}

{\textbf{Faculdade de Ciências da Universidade do Porto}},
Porto, Portugal

\GapNoBreak
\BulletItem
Undergradute Research Project
\hfill
\DatestampYMD{2014}{02}{15} --
\DatestampYMD{2014}{07}{15}
\begin{detail}
\SubBulletItem
Project:
Study of the mizimization function in the ARES+MOOG procedure
\SubBulletItem
Supervisor:
Dr. Sérgio Sousa
\SubBulletItem
Research areas: Spectroscopic Parameters, iron line abundances, Python
\end{detail}

\GapNoBreak
\BulletItem
PEEC (Extra-Curricular Research Project)
\hfill
\DatestampYMD{2013}{10}{15} --
\DatestampYMD{2015}{07}{15}
\begin{detail}
\SubBulletItem
Project:
Testing the applicability of using titanium line abundances to determine spectroscopic stellar parameters, especially surface gravity, using the ARES and MOOG tools
\SubBulletItem
Supervisor:
Dr. Sérgio Sousa
\SubBulletItem
Research areas: Spectroscopic Parameters, titanium line abundances, Python
\end{detail}

%%%%%%%%%%%%%%%%%%
%% PUBLICATIONS %%
%%%%%%%%%%%%%%%%%%

\section
{Publications}
{Publications}
{PDF:Publications}

% \subsection
% {Journals}
% {Journals}
% {PDF:Journals}

% \GapNoBreak
% \NumberedItem{[11]}
% \href{http://www.example.com/my-paper-doi-5}
% {\underline{J.~Doe}, J.~Citizen, and A.~Yone,
% ``On lasers and climate change,''
% \textit{Journal of Science},
% vol.~89,
% no.~2,
% pp.~4123--4133,
% \DatestampYM{2008}{02}.}

% % Note the use of {\CharSpace} for aligning shorter numbers.
% \Gap
% \NumberedItem{{\CharSpace}[1]}
% \href{http://www.example.com/my-paper-doi-4}
% {\underline{J.~Doe} and J.~Citizen,
% ``Measuring the extent of climate change,''
% \textit{Global Scientific Journal},
% vol.~12,
% no.~4,
% pp.~330--352,
% \DatestampYM{2006}{12}.}

% \BigGap
\subsection
{Conferences}
{Conferences}
{PDF:Conferences}

\GapNoBreak
\NumberedItem{1.}
\href{}
{\underline{Pereira L.F.R.}, Faria J.P.S., and Monteiro M.J.P.F.G., 
\DatestampY{2016}, 
SIGS - Seismic Inferences for Glitches in Stars,
in \textit{Seismology of the Sun and the Distant Stars 2016},
submitted.}

% \Gap
% \NumberedItem{[10]}
% \href{http://www.example.com/my-paper-doi-2}
% {A.~Yone and \underline{J.~Doe},
% ``Climate change and general relativity,''
% in \textit{Proceedings of the International Astronomical Conference},
% Sydney, Australia,
% \DatestampYM{2006}{8}.}

% % Note the use of {\CharSpace} for aligning shorter numbers.
% \Gap
% \NumberedItem{{\CharSpace}[1]}
% \href{http://www.example.com/my-paper-doi-1}
% {\underline{J.~Doe} and J.~Citizen,
% ``Measuring the extent of climate change,''
% in \textit{Proceedings of the International Climate Change Conference},
% London, UK,
% \DatestampYM{2005}{11}.}

%%%%%%%%%%%%%%%%%%%%%
%% ACADEMIC AWARDS %%
%%%%%%%%%%%%%%%%%%%%%

\section
{Grants}
{Grants}
{PDF:Grants}

Development of automatic tools for measuring acoustic glitches in seismic data of solar-type stars
\hfill
\DatestampYMD{2015}{10}{01} --
\DatestampYMD{2016}{06}{30}
\BulletItem
Project: SPACEINN (FP7-SPACE-2012-312844)
\BulletItem
Supervisor: Mário João Monteiro
\BulletItem
Institution: Centro de Astrofísica da Universidade do Porto (CAUP)
\BulletItem
Grant: \textasciitilde 7000 EUR


\BigGap
​Characterizing and modelling red giants
\hfill
\DatestampYMD{2017}{01}{01} --
\DatestampYMD{2017}{12}{31}
\BulletItem
Project: ​CIAAUP-09/2016-BI
\BulletItem
Supervisor: Margarida Cunha
\BulletItem
Institution: Centro de Astrofísica da Universidade do Porto (CAUP)
\BulletItem
Grant: \textasciitilde 12000 EUR

%%%%%%%%%%%%%%%%%%%%%%%%%%%%%%%%%%%%%%%%%%%%
%% SCIENTIFIC ACTIVITIES                  %%
%%%%%%%%%%%%%%%%%%%%%%%%%%%%%%%%%%%%%%%%%%%%

\section
{Scientific Activities}
{Scientific Activities}
{PDF:ScientificActivities}

\href{http://www.iastro.pt/research/conferences/spacetk16/}
{\textbf{Seismology of the Sun and the Distant Stars 2016, Joint TASC2 \& KASC9 Workshop – SPACEINN \& HELAS8 Conference}},
Azores, Portugal
\hfill
\DatestampYMD{2016}{07}{11}
\GapNoBreak
\BulletItem
Poster presentation describing some of the results of the research for the M.Sc. dissertation.


\GapNoBreak
\href{http://www.iastro.pt/research/conferences/faial2016/}
{\textbf{IVth Azores International Advanced School in Space Sciences}},
Azores, Portugal
\hfill
\DatestampYMD{2016}{07}{17}
\GapNoBreak
\BulletItem
Participant in the Summer School and member of the Local Organising Committee.

\GapNoBreak
\href{http://sac.au.dk/currently/tess-data-for-asteroseismology-workshop/}
{\textbf{1st TESS Data for Asteroseismology workshop - T'DA 1}},
Birmingham, England
\hfill
\DatestampYMD{2016}{10}{30}
\GapNoBreak
\BulletItem
Participant in the workshop.

\GapNoBreak
\href{http://sac.au.dk/currently/2nd-tess-data-for-asteroseismology-workshop/}
{\textbf{2nd TESS Data for Asteroseismology workshop - T'DA 2}},
Aarhus, Denmark
\hfill
\DatestampYMD{2017}{04}{18}
\GapNoBreak
\BulletItem
Will participate in the workshop.


%%%%%%%%%%%%%%%%%%%%%%%%%%%%%%%%%%%%%%%%%%%%
%% SCIENTIFIC AFFILIATIONS                %%
%%%%%%%%%%%%%%%%%%%%%%%%%%%%%%%%%%%%%%%%%%%%

\section
{Scientific Affiliations}
{Scientific Affiliations}
{PDF:ScientificAffiliations}

\href{http://www.sp-astronomia.pt/}
{\textbf{Sociedade Portuguesa de Astronomia}},
Portugal

\GapNoBreak
\BulletItem
Member
\hfill
\DatestampY{2012} --
Present

%%%%%%%%%%%%%%%
%% LANGUAGES %%
%%%%%%%%%%%%%%%

\section
{Languages}
{Languages}
{PDF:Languages}

\BulletItem
Portuguese: Native language.

\GapNoBreak
\BulletItem
English: Fluent (speaking, reading, writing), C1 Level.

\GapNoBreak
\BulletItem
Spanish: Intermediate (reading, speaking); basic (writing).

%%%%%%%%%%%%
%% SKILLS %%
%%%%%%%%%%%%

\section
{Computer Skills}
{Computer Skills}
{PDF:Computer Skills}

\BulletItem
Basic Knowledge: Java, MySQL, C, Linux, R

\GapNoBreak
\BulletItem
Intermediate Knowledge: {\LaTeX}, Python, MATLAB, Fortran

%%%%%%%%%%%%%%%
%% INTERESTS %%
%%%%%%%%%%%%%%%

\section
{Interests}
{Interests}
{PDF:Interests}

Computer Science, Cinema, Basketball, Voleyball
\end{body}
\label{LastPage}~
\end{document}
