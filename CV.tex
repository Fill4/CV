% !TEX TS-program = xelatex
% !TEX encoding = UTF-8

%%%%%%%%%%%%%%%%%%%%%%%%%%%%%%%%%%%%%%%%%%%%%%%%%%%%%%%%%%%%%%%%%
%% SIMPLE-RESUME-CV
%% <https://github.com/zachscrivena/simple-resume-cv>
%% This is free and unencumbered software released into the
%% public domain; see <http://unlicense.org> for details.
%%%%%%%%%%%%%%%%%%%%%%%%%%%%%%%%%%%%%%%%%%%%%%%%%%%%%%%%%%%%%%%%%

%%%%%%%%%%%%%%%%%%%%%%%%%%%%%%%%%%%%%%%%%%%%%%%%%%%%%%%%%%%%%%%%%
%% INSTRUCTIONS FOR COMPILING THIS DOCUMENT ("CV.tex")
%% TeX ---(XeLaTeX)---> PDF:
%%
%% Method 1: Use latexmk for fully automated document generation:
%%   latexmk -xelatex "CV.tex"
%%   (add the -pvc switch to automatically recompile on changes)
%%
%% Method 2: Use XeLaTeX directly:
%%   xelatex "CV.tex"
%%   (run multiple times to resolve cross-references if needed)
%%%%%%%%%%%%%%%%%%%%%%%%%%%%%%%%%%%%%%%%%%%%%%%%%%%%%%%%%%%%%%%%%

% \documentclass[a4paper,10pt,oneside]{article}
\documentclass[letterpaper,10pt,oneside]{article}

%%%%%%%%%%%%%%%%%%%%%%%%%%%%%%%%%%%%%%%%%%%%%%%%%%%%%%%%%%%%%%%%%
%% TYPESETTING OPTIONS.
%%%%%%%%%%%%%%%%%%%%%%%%%%%%%%%%%%%%%%%%%%%%%%%%%%%%%%%%%%%%%%%%%

\newcommand{\TypesetInNonStopMode}{1}
\newcommand{\TypesetInDraftMode}{0}

%%%%%%%%%%%%%%%%%%%%%%%%%%%%%%%%%%%%%%%%%%%%%%%%%%%%%%%%%%%%%%%%%
%% PREAMBLE.
%%%%%%%%%%%%%%%%%%%%%%%%%%%%%%%%%%%%%%%%%%%%%%%%%%%%%%%%%%%%%%%%%

%%%%%%%%%%%%%%%%%%%%%%%%%%%%%%%%%%%%%%%%%%%%%%%%%%%%%%%%%%%%%%%%%
%% SIMPLE-RESUME-CV
%% <https://github.com/zachscrivena/simple-resume-cv>
%% This is free and unencumbered software released into the
%% public domain; see <http://unlicense.org> for details.
%%%%%%%%%%%%%%%%%%%%%%%%%%%%%%%%%%%%%%%%%%%%%%%%%%%%%%%%%%%%%%%%%

% Run in non-stop mode.
\ifnum\TypesetInNonStopMode=1
\nonstopmode
\fi

% Geometry package for page margins.
\usepackage[
left=0.70in,
right=0.70in,
top=0.60in,
bottom=0.45in,
nohead,
includefoot]{geometry}

% PDF settings and properties.
\usepackage{hyperref}

% Long table for page layout.
\usepackage{longtable}

% Hyphenation: Disabled.
\usepackage[none]{hyphenat}

% Colors.
\usepackage[usenames]{color}
% \definecolor{MyDarkBlue}{RGB}{0,90,160}
% {\color{MyDarkBlue}This text is dark blue}

% Current date and time.
\usepackage[yyyymmdd,24hr]{datetime}
\renewcommand{\dateseparator}{-}
\settimeformat{xxivtime}
% {\today}~{\currenttime}

% Timestamp.
\newcommand{\Timestamp}{{\yyyymmdddate\today}~{\currenttime}}

% Abbreviations for months.
\newcommand{\LongMonth}[1]{%
\ifcase#1\relax
\or January%
\or February%
\or March%
\or April%
\or May%
\or June%
\or July%
\or August%
\or September%
\or October%
\or November%
\or December%
\fi}
\newcommand{\ShortMonth}[1]{%
\ifcase#1\relax
\or Jan%
\or Feb%
\or Mar%
\or Apr%
\or May%
\or Jun%
\or Jul%
\or Aug%
\or Sep%
\or Oct%
\or Nov%
\or Dec%
\fi}

% Select datestamp format.
\def\DatestampFormatSelection{2}

% Datestamp format: {yyyy}{MM}{dd} ---> yyyy-MM-dd (e.g., 2010-12-31).
\ifnum\DatestampFormatSelection=1
\newcommand{\DatestampYMD}[3]{\mbox{#1-#2-#3}}
\newcommand{\DatestampYM}[2]{\mbox{#1-#2}}
\newcommand{\DatestampY}[1]{#1}
\fi

% Datestamp format: {yyyy}{MM}{dd} ---> MMM yyyy (e.g., Dec 2010).
\ifnum\DatestampFormatSelection=2
\newcommand{\DatestampYMD}[3]{\mbox{\ShortMonth{#2} #1}}
\newcommand{\DatestampYM}[2]{\mbox{\ShortMonth{#2} #1}}
\newcommand{\DatestampY}[1]{#1}
\fi

% Datestamp format: {yyyy}{MM}{dd} ---> MMMM yyyy (e.g., December 2010).
\ifnum\DatestampFormatSelection=3
\newcommand{\DatestampYMD}[3]{\mbox{\LongMonth{#2} #1}}
\newcommand{\DatestampYM}[2]{\mbox{\LongMonth{#2} #1}}
\newcommand{\DatestampY}[1]{#1}
\fi

% Datestamp format: {yyyy}{MM}{dd} ---> yyyy (e.g., 2010).
\ifnum\DatestampFormatSelection=4
\newcommand{\DatestampYMD}[3]{#1}
\newcommand{\DatestampYM}[2]{#1}
\newcommand{\DatestampY}[1]{#1}
\fi

% XeLaTeX packages.
\usepackage{fontspec}
\defaultfontfeatures{Ligatures=TeX}
\usepackage{xunicode}
\usepackage{xltxtra}

% Font: Use "Tinos" as the main typeface (\textnormal{}, \normalfont).
% The "Tinos" fonts are released under the Apache License Version 2.0,
% and can be downloaded for free at <http://www.fontsquirrel.com/fonts/tinos>.
% Symbol table: <http://www.fileformat.info/info/unicode/font/tinos/grid.htm>
\setmainfont
[Path=./Fonts/Tinos/,
ItalicFont=Tinos-Italic,
BoldFont=Tinos-Bold,
BoldItalicFont=Tinos-BoldItalic]
{Tinos-Regular.ttf}

% Secondary font: "GNU FreeFont".
% The "GNU FreeFont" fonts are released under the
% GNU General Public License Version 3, and can be downloaded
% for free at <https://savannah.gnu.org/projects/freefont/>.
\newcommand{\UseSecondaryFont}{\fontspec
[Path=./Fonts/GNUFreeFont/,
ItalicFont=FreeSerifItalic,
BoldFont=FreeSerifBold,
BoldItalicFont=FreeSerifBoldItalic]
{FreeSerif.otf}}

% Sans-serif font: Changed to "Tinos".
\renewcommand{\sffamily}{\rmfamily}

% Typewriter (monospace) font: Changed to "Tinos".
\renewcommand{\ttfamily}{\rmfamily}

% Small caps font: Changed to "Tinos".
\renewcommand{\scshape}{\rmfamily}
%\renewcommand{\textsc}[1]{\textbf{\MakeUppercase{#1}}}

% Font styles.
\newcommand{\UseHeadingFont}{\normalfont}
\newcommand{\UseHeaderFooterFont}{\UseHeadingFont\fontsize{8.2pt}{9.5pt}\selectfont}
\newcommand{\UseNoteFont}{\UseHeadingFont\fontsize{8pt}{9.6pt}\selectfont}
\newcommand{\UseTitleFont}{\UseHeadingFont\fontsize{28pt}{33.6pt}\selectfont\bfseries}
\newcommand{\UseSubTitleFont}{\normalfont\fontsize{8.6pt}{10.3pt}\selectfont}
\newcommand{\UseSectionFont}{\UseHeadingFont\fontsize{9pt}{11pt}\selectfont\bfseries}
\newcommand{\UseSubSectionFont}{\UseHeadingFont\fontsize{8.6pt}{10.3pt}\selectfont\bfseries}
\newcommand{\UseDetailFont}{\normalfont\fontsize{8.6pt}{10.3pt}\selectfont}

% Symbols (unicode).
\newcommand{\BulletSymbol}{{\normalfont\fontsize{6.5}{7.8}\selectfont\raisebox{0.17em}{\char"25A0}}}
\newcommand{\SubBulletSymbol}{{\normalfont\fontsize{6}{7.2}\selectfont\raisebox{0.17em}{\char"25CF}}}
\newcommand{\TildeSymbol}{{\normalfont\char"007E}}

% Headers and footers: Blank header, page number in footer.
\ifnum\TypesetInDraftMode=0
\newcommand{\HeaderText}{}
\newcommand{\FooterText}{\UseHeaderFooterFont\hfill%
{Page}~{\thepage}~of~\pageref{LastPage}%
\hfill}
\else
\newcommand{\HeaderText}{}
\newcommand{\FooterText}{\UseHeaderFooterFont%
\hphantom{DRAFT~\Timestamp}\hfill%
{Page}~{\thepage}~of~\pageref{LastPage}%
\hfill{\color{red}DRAFT~\Timestamp}}
\fi

\makeatletter
\def\ps@plain{%
\def\@oddhead{\HeaderText}%
\def\@evenhead{\HeaderText}%
\def\@oddfoot{\FooterText}%
\def\@evenfoot{\FooterText}}
\makeatother

\pagestyle{plain}

% Paragraph style.
\setlength{\parindent}{0in} % No indentation at the beginning of each paragraph.
\setlength{\parskip}{0in} % No vertical space between paragraphs.

% Footnotes: Use symbols instead of numbers for labels.
\renewcommand{\thefootnote}{\fnsymbol{footnote}}

% Macro: title (name).
\renewcommand{\title}[1]{%
\pdfbookmark[1]{#1}{#1}%
\par\begin{center}%
\par\UseTitleFont%
{#1}%
\par\end{center}%
\par\vspace{-1.75em}\par}

% Macro: subtitle (personal information below name).
\newenvironment{subtitle}
{\par\begin{center}%
\par\UseSubTitleFont}
{\par\end{center}\par}

% Macro: body (rest of the document).
\newenvironment{body}
{\par\vspace{-1em}\par
\begin{longtable}{p{0.15\textwidth}p{0.80\textwidth}}}
{\par\end{longtable}\par}

% Macro: section (new section for Education, Research Experience, etc.).
\renewcommand{\section}[3]{\\[-1em]\pdfbookmark[2]{#2}{#3}\\%
{\UseSectionFont\raggedright\MakeUppercase{#1}}%
&}

% Macro: subsection.
\renewcommand{\subsection}[3]{\par~\vskip-\baselineskip%
\pdfbookmark[3]{#2}{#3}\par%
{\UseSubSectionFont\raggedright\MakeUppercase{#1}}%
\vspace{0.225\baselineskip}}

% Macro: BigGap, BigGapNoBreak (big vertical gap between items in the same section).
\newcommand{\BigGap}{\\[-1.75mm]~&}
\newcommand{\BigGapNoBreak}{\par\vspace{2.45mm}\par}

% Macro: Gap, GapNoBreak (vertical gap between items in the same section).
\newcommand{\Gap}{\\[-3.5mm]~&}
\newcommand{\GapNoBreak}{\par\vspace{0.7mm}\par}

% Macro: detail (text in smaller font under an item).
\newenvironment{detail}
{\par\begingroup\UseDetailFont}
{\par\endgroup\par}

% Macro: BulletItem.
\newsavebox{\BulletItemIndentation}
\newlength{\BulletItemIndentationWidth}
\newcommand{\BulletItem}{\par%
\savebox{\BulletItemIndentation}{\hspace{1.5mm}\BulletSymbol\hspace{1.25mm}}%
\settowidth{\BulletItemIndentationWidth}{\usebox{\BulletItemIndentation}}%
\noindent\hangafter=1\hangindent=\BulletItemIndentationWidth\ignorespaces%
\usebox{\BulletItemIndentation}\ignorespaces}

% Macro: SubBulletItem.
\newsavebox{\SubBulletItemIndentation}
\newlength{\SubBulletItemIndentationWidth}
\newcommand{\SubBulletItem}{\par%
\savebox{\SubBulletItemIndentation}{\hspace{5.6mm}\SubBulletSymbol\hspace{1.25mm}}%
\settowidth{\SubBulletItemIndentationWidth}{\usebox{\SubBulletItemIndentation}}%
\noindent\hangafter=1\hangindent=\SubBulletItemIndentationWidth\ignorespaces%
\usebox{\SubBulletItemIndentation}\ignorespaces}

% Macro: Item.
\newsavebox{\ItemIndentation}
\newlength{\ItemIndentationWidth}
\newcommand{\Item}{\par%
\savebox{\ItemIndentation}{\hphantom{\hspace{1.5mm}\BulletSymbol\hspace{1.25mm}}}%
\settowidth{\ItemIndentationWidth}{\usebox{\ItemIndentation}}%
\noindent\hangafter=1\hangindent=\ItemIndentationWidth\ignorespaces%
\usebox{\ItemIndentation}\ignorespaces}

% Macro: SubItem.
\newsavebox{\SubItemIndentation}
\newlength{\SubItemIndentationWidth}
\newcommand{\SubItem}{\par%
\savebox{\SubItemIndentation}{\hphantom{\hspace{1.5mm}\BulletSymbol\hspace{1.25mm}}}%
\settowidth{\SubItemIndentationWidth}{\usebox{\SubItemIndentation}}%
\noindent\hangafter=1\hangindent=\SubItemIndentationWidth\ignorespaces%
\usebox{\SubItemIndentation}\ignorespaces}

% Macro: NumberedItem.
\newsavebox{\NumberedItemIndentation}
\newlength{\NumberedItemIndentationWidth}
\newcommand{\NumberedItem}[1]{\par%
\savebox{\NumberedItemIndentation}{{#1}\hspace{2.3mm}}%
\settowidth{\NumberedItemIndentationWidth}{\usebox{\NumberedItemIndentation}}%
\noindent\hangafter=1\hangindent=\NumberedItemIndentationWidth\ignorespaces%
\usebox{\NumberedItemIndentation}\ignorespaces}

% Macro: CharSpace (for aligning single-digit numbers).
\newlength{\CharWidth}
\newcommand{\CharSpace}{\settowidth{\CharWidth}{8}\hspace{\CharWidth}}

% Macro: hide.
\newcommand{\hide}[1]{}


% CV Info (to be customized).
\newcommand{\CVAuthor}{Filipe Pereira}
\newcommand{\CVTitle}{Curriculum Vit\ae}
\newcommand{\CVNote}{CV compiled on {\today} for Acme Corporation}
\newcommand{\CVWebpage}{http://www.example.com/johndoe}

% PDF settings and properties.
\hypersetup{
pdftitle={\CVTitle},
pdfauthor={\CVAuthor},
pdfsubject={\CVWebpage},
pdfcreator={XeLaTeX},
pdfproducer={},
pdfkeywords={},
pdfpagemode={},
bookmarks=true,
unicode=true,
bookmarksopen=true,
pdfstartview=FitH,
pdfpagelayout=OneColumn,
pdfpagemode=UseOutlines,
hidelinks,
breaklinks}

% Shorthand.
\newcommand{\CodeCommand}[1]{\mbox{\textbf{\textbackslash{#1}}}}

%%%%%%%%%%%%%%%%%%%%%%%%%%%%%%%%%%%%%%%%%%%%%%%%%%%%%%%%%%%%%%%%%
%% ACTUAL DOCUMENT.
%%%%%%%%%%%%%%%%%%%%%%%%%%%%%%%%%%%%%%%%%%%%%%%%%%%%%%%%%%%%%%%%%

\begin{document}

%%%%%%%%%%%%%%%
% TITLE BLOCK %
%%%%%%%%%%%%%%%

\title{\CVAuthor}
\subtitle{\CVTitle}

\begin{subtitle}
%{Rua Dr. Orlando de Oliveira, nº11, 2ºDto, 3800-004, Aveiro, Portugal}
\par
\href{mailto:filipe.pereira@astro.up.pt}
{filipe.pereira@astro.up.pt}
%\,\SubBulletSymbol\,
%+351 913 231 818
%\,\SubBulletSymbol\,
%\href{\CVWebpage}
%{\CVWebpage}
\end{subtitle}

\begin{body}

%%%%%%%%%%%%%%%
%% EDUCATION %%
%%%%%%%%%%%%%%%

\section
{Education}
{Education}
{PDF:Education}

{\textbf{Faculty of Sciences of the University of Porto}},
Porto, Portugal

\GapNoBreak
\BulletItem
Doctoral Programme in Astronomy
\hfill
\DatestampYMD{2016}{10}{01} --
Ongoing
\begin{detail}
\SubBulletItem
Completed courses: Exoplanets; Asteroseismology; Jets
\end{detail}

\GapNoBreak
\BulletItem
M.Sc. in Astronomy
\hfill
\DatestampYMD{2014}{10}{15} --
\DatestampYMD{2016}{10}{15}

%\begin{detail}
%\SubBulletItem
%Thesis: Development of automatic tools for measuring acoustic glitches in seismic data of solar-type stars
%\SubBulletItem
%Supervisor:
%Professor Mário João Monteiro
%\SubBulletItem
%Research areas: Asteroseismology, acoustic glicthes, Fortran
%\end{detail}

\GapNoBreak
\BulletItem
Bachelor's Degree in Astronomy
\hfill
\DatestampYMD{2011}{09}{15} --
\DatestampYMD{2014}{09}{15}

%%%%%%%%%%%%%%%%%%%%%%%%%%%%%%%%%
%% OTHER CURRICULAR ACTIVITIES %%
%%%%%%%%%%%%%%%%%%%%%%%%%%%%%%%%%

\section
{Other Curricular Activities}
{Other Curricular Activities}
{PDF:OtherCurricularActivities}

\href{http://www.iastro.pt/research/conferences/faial2016/}
{\textbf{IVth Azores International Advanced School in Space Sciences}},
Azores, Portugal
\hfill
\DatestampYMD{2016}{07}{17}
\GapNoBreak
\BulletItem
Completed various courses totaling 49 hours.
\BulletItem
Included courses in: Stellar Modelling, Theory of Stellar Oscillations, Data Analysis in Asteroseismology, Exoplanetary Science, Analysis of Photometric Time Series

\BigGap
{\textbf{Machine Learning Course from Stanford University}}
\hfill
\DatestampYMD{2015}{09}{15} --
\DatestampYMD{2015}{12}{07}


%%%%%%%%%%%%%%%%%%%%%%%%%%%%
%% ACADEMIC RESEARCH WORK %%
%%%%%%%%%%%%%%%%%%%%%%%%%%%%

\section
{Academic Research Work}
{Academic Research Work}
{PDF:AcademicResearchWork}

{\textbf{Msc Dissertation}}
\hfill
\DatestampYMD{2016}{10}{15} --
\DatestampYMD{2015}{10}{01}
\begin{detail}
\SubBulletItem
Thesis: Development of automatic tools for measuring acoustic glitches in seismic data of solar-type stars
\SubBulletItem
Supervisor:
Dr. Mário João Monteiro
\SubBulletItem
Research areas: Asteroseismology, acoustic glicthes, Fortran
\end{detail}

\GapNoBreak
{\textbf{Undergradute Research Project}}
\hfill
\DatestampYMD{2014}{02}{15} --
\DatestampYMD{2014}{07}{15}
\begin{detail}
\SubBulletItem
Project:
Study of the mizimization function in the ARES+MOOG procedure
\SubBulletItem
Supervisor:
Dr. Sérgio Sousa
\SubBulletItem
Research areas: Spectroscopic Parameters, iron line abundances, Python
\end{detail}


%%%%%%%%%%%%%%%%%%%%%%%%%
%% RESEARCH EXPERIENCE %%
%%%%%%%%%%%%%%%%%%%%%%%%%

%\section
%{Research Experience}
%{Research Experience}
%{PDF:ResearchExperience}

%%%%%%%%%%%%%%%%%%%%%%%%%%%%%%%%%%%%%%%%%%%%
%% SCIENTIFIC ACTIVITIES                  %%
%%%%%%%%%%%%%%%%%%%%%%%%%%%%%%%%%%%%%%%%%%%%

\section
{Research Activities}
{Research Activities}
{PDF:ResearchActivities}

{\textbf{Research Internship}}
\hfill
\DatestampYMD{2017}{10}{15} -- Ongoing
\begin{detail}
\SubBulletItem
Project:
​Characterizing and modelling red giants
\SubBulletItem
Supervisor:
Dr. Margarida Cunha
\SubBulletItem
Research areas: gaussian processes, grid-based modelling, red-giant stars
\end{detail}

\BigGap
{\textbf{PEEC (Extra-Curricular Research Project)}}
\hfill
\DatestampYMD{2013}{10}{15} --
\DatestampYMD{2015}{07}{15}
\begin{detail}
\SubBulletItem
Project:
Testing the applicability of using titanium line abundances to determine spectroscopic stellar parameters, especially surface gravity, using the ARES and MOOG tools
\SubBulletItem
Supervisor:
Dr. Sérgio Sousa
\SubBulletItem
Research areas: Spectroscopic Parameters, titanium line abundances, Python
\end{detail}

\BigGap
\href{http://www.iastro.pt/research/conferences/spacetk16/}
{\textbf{Seismology of the Sun and the Distant Stars 2016, Joint TASC2 \& KASC9 Workshop – SPACEINN \& HELAS8 Conference}},
Azores, Portugal
\hfill
\DatestampYMD{2016}{07}{11}

\BigGap
\href{http://sac.au.dk/currently/tess-data-for-asteroseismology-workshop/}
{\textbf{1st TESS Data for Asteroseismology workshop - T'DA 1}},
Birmingham, England
\hfill
\DatestampYMD{2016}{10}{30}

\BigGap
\href{http://sac.au.dk/currently/2nd-tess-data-for-asteroseismology-workshop/}
{\textbf{2nd TESS Data for Asteroseismology workshop - T'DA 2}},
Aarhus, Denmark
\hfill
\DatestampYMD{2017}{04}{18}
\GapNoBreak
\BulletItem
Will participate in the workshop.

\BigGap
\textbf{Member of the IA Stars \& Planets Research Team}
\GapNoBreak
\BulletItem
Participate in bi-weekly meetings with the Asteroseismology Team
\GapNoBreak
\BulletItem
Participate in monthly meetings with the Exoplanets Team

\BigGap
\textbf{Member of CAUP (Center for Astrophysics of the University of Porto)}
\GapNoBreak
\BulletItem
Attend regular seminar in various Astronomy topics presented by both local and visiting researchers

%%%%%%%%%%%%%%%%%%
%% PUBLICATIONS %%
%%%%%%%%%%%%%%%%%%

\section
{Scientific Results}
{Scientific Results}
{PDF:ScientificResults}

% \subsection
% {Journals}
% {Journals}
% {PDF:Journals}

% \GapNoBreak
% \NumberedItem{[11]}
% \href{http://www.example.com/my-paper-doi-5}
% {\underline{J.~Doe}, J.~Citizen, and A.~Yone,
% ``On lasers and climate change,''
% \textit{Journal of Science},
% vol.~89,
% no.~2,
% pp.~4123--4133,
% \DatestampYM{2008}{02}.}

% % Note the use of {\CharSpace} for aligning shorter numbers.
% \Gap
% \NumberedItem{{\CharSpace}[1]}
% \href{http://www.example.com/my-paper-doi-4}
% {\underline{J.~Doe} and J.~Citizen,
% ``Measuring the extent of climate change,''
% \textit{Global Scientific Journal},
% vol.~12,
% no.~4,
% pp.~330--352,
% \DatestampYM{2006}{12}.}

% \BigGap
\subsection
{Proceedings}
{Proceedings}
{PDF:Proceedings}

\GapNoBreak
\NumberedItem{1.}
\href{}
{\underline{Pereira L.F.R.}, Faria J.P.S., and Monteiro M.J.P.F.G., 
\DatestampY{2016}, 
SIGS - Seismic Inferences for Glitches in Stars,
in Proceedings for the \textit{Seismology of the Sun and the Distant Stars 2016} Conference, \href{http://arxiv.org/abs/1703.04828} {arXiv}, submitted.}

\BigGap
\subsection
{Posters}
{Posters}
{PDF:Posters}

\NumberedItem{1.}
{\underline{Pereira L.F.R.}, Faria J.P.S., and Monteiro M.J.P.F.G., 
\DatestampY{2016}, 
SIGS - Seismic Inferences for Glitches in Stars,
in \textit{Seismology of the Sun and the Distant Stars 2016} Conference.}

\BigGap
\subsection
{Reports}
{Reports}
{PDF:Reports}

\NumberedItem{1.}
Msc Grant Final Report
\NumberedItem{2.}
PEEC Project Report

% \Gap
% \NumberedItem{[10]}
% \href{http://www.example.com/my-paper-doi-2}
% {A.~Yone and \underline{J.~Doe},
% ``Climate change and general relativity,''
% in \textit{Proceedings of the International Astronomical Conference},
% Sydney, Australia,
% \DatestampYM{2006}{8}.}

% % Note the use of {\CharSpace} for aligning shorter numbers.
% \Gap
% \NumberedItem{{\CharSpace}[1]}
% \href{http://www.example.com/my-paper-doi-1}
% {\underline{J.~Doe} and J.~Citizen,
% ``Measuring the extent of climate change,''
% in \textit{Proceedings of the International Climate Change Conference},
% London, UK,
% \DatestampYM{2005}{11}.}

%%%%%%%%%%%%%%%%%%%%%
%% ACADEMIC AWARDS %%
%%%%%%%%%%%%%%%%%%%%%

\section
{Grants}
{Grants}
{PDF:Grants}

Development of automatic tools for measuring acoustic glitches in seismic data of solar-type stars
\hfill
\DatestampYMD{2015}{10}{01} --
\DatestampYMD{2016}{06}{30}
\BulletItem
Project: SPACEINN (FP7-SPACE-2012-312844)
\BulletItem
Supervisor: Dr. Mário João Monteiro
\BulletItem
Institution: Centro de Astrofísica da Universidade do Porto (CAUP)
\BulletItem
Grant: \textasciitilde 7000 EUR


\BigGap
​Characterizing and modelling red giants
\hfill
\DatestampYMD{2017}{01}{01} --
\DatestampYMD{2017}{12}{31}
\BulletItem
Project: ​CIAAUP-09/2016-BI
\BulletItem
Supervisor: Dr. Margarida Cunha
\BulletItem
Institution: Centro de Astrofísica da Universidade do Porto (CAUP)
\BulletItem
Grant: \textasciitilde 12000 EUR


%%%%%%%%%%%%%%%%%%%%%%%%%%%%%%%%%%%%%%%%%%%%
%% SCIENTIFIC AFFILIATIONS                %%
%%%%%%%%%%%%%%%%%%%%%%%%%%%%%%%%%%%%%%%%%%%%

\section
{Scientific Affiliations}
{Scientific Affiliations}
{PDF:ScientificAffiliations}

\href{http://www.sp-astronomia.pt/}
{\textbf{Sociedade Portuguesa de Astronomia}},
Portugal

\GapNoBreak
\BulletItem
Member
\hfill
\DatestampY{2012} --
Present

%%%%%%%%%%%%%%%
%% LANGUAGES %%
%%%%%%%%%%%%%%%

\section
{Languages}
{Languages}
{PDF:Languages}

\BulletItem
Portuguese: Native language.

\GapNoBreak
\BulletItem
English: Fluent (speaking, reading, writing), C1 Level.

\GapNoBreak
\BulletItem
Spanish: Intermediate (reading, speaking); basic (writing).

%%%%%%%%%%%%
%% SKILLS %%
%%%%%%%%%%%%

\section
{Computer Skills}
{Computer Skills}
{PDF:Computer Skills}

\BulletItem
Basic Knowledge: Java, MySQL, C, Linux, R

\GapNoBreak
\BulletItem
Intermediate Knowledge: {\LaTeX}, Python, MATLAB, Fortran

%%%%%%%%%%%%%%%
%% INTERESTS %%
%%%%%%%%%%%%%%%

\section
{Interests}
{Interests}
{PDF:Interests}

Computer Science, Cinema, Basketball, Voleyball
\end{body}
\label{LastPage}~
\end{document}
